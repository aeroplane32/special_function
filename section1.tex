\documentclass[main.tex]{subfiles}

\begin{document}

\section{标准实数集的结构}

这里强调接下来构造的实数集是“标准”实数集,即构造的实数集遵从牛顿/莱布尼兹-柯西-魏尔斯特拉斯的路线,其拓扑结构采用标准拓扑,有别于鲁滨逊非标准分析的构造,此时\textbf{假设有理数集已经构造好了},而每个有理数都形如\(p/q\)(\(p,q\)为整数且\(q \neq 0\)). 从有理数构造实数有几种方法,例如柯西数列、戴德金分割和公理化定义,接下来使用柯西数列构造,因为完整地定义实数的十进制表示法需要用到柯西数列.

%我们总是希望用尽可能少的数学对象来描述尽可能多的“数学现象”。希尔伯特时代的数学家们尝试寻找数学大厦的“大一统理论”,即一套能够推导出所有数学定理的公理系统,尽管哥德尔不完备性定理宣告这种努力是白费的。目前主流数学界认为集合论和谓词演算是整个数学大厦的基础,冯·诺依曼把关于“\(\in\)”的良序有限集定义为自然数,皮亚诺定义出了自然数的加法、乘法和乘方。然后为了让加法完备\footnote{这里的“完备”解释起来比较复杂,可以大致理解为扩展其定义域}而出现了整数,为了让乘法完备而出现了有理数。但有理数和实数之间有一条巨大的鸿沟,让乘方完备确实可以再次扩充数集,但也只能得到代数数,实际上非代数数比代数数还要多的多。

\subsection{从有理数构造实数}

有理数集\(\mathbb{Q}\)上的大小关系(序关系)是稠密的,这是自然数集\(\mathbb{N}\)和整数集\(\mathbb{Z}\)都不具备的性质.

\begin{definition}{序关系的稠密性}
    设\(\leq\)为定义在集合\(U\)上的序关系,在\(U\)任取两个元素\(a,b\),若存在另外一个元素\(c\),使得\(a\leq c \leq b\),则称该序关系是\uline{稠密的}(dense).
\end{definition}

有理数集上的序关系就是稠密的,也就是说任取两个有理数,无论它们相差多小,只要不相等,都可以在它们之间找到有理数,它们的算术平均就是一个例子. 接下来任取两个不等的有理数\(a,b\),然后取它们之间的有理数\(b_1\),再取\(a\)和\(b_1\)之间的有理数\(b_2\),再取\(a\)和\(b_2\)之间的有理数\(b_3\)……如此操作下去,有理数列\(\{b_n\}\)必然会收敛于某一点(柯西审敛原理),而这一点不一定是有理数,比如构造两个有理数列\(\{a_n\}_{n=0}^{\infty}\)和\(\{b_n\}_{n=0}^{\infty}\):
\begin{eqnarray*}
    a_n=\frac{k}{10^n}, \quad\left(\frac{k}{10^n}\right)^2<2<\left(\frac{k+1}{10^n}\right)^2, \quad k\in\mathbb{N} \\
    b_n=\frac{k+1}{10^n}, \quad\left(\frac{k}{10^n}\right)^2<2<\left(\frac{k+1}{10^n}\right)^2, \quad k\in\mathbb{N}
\end{eqnarray*}
两个数列一个有界严格递增,一个有界严格递减,可以列出前几项:
\newline
\begin{align*}
a_0 &= 1 & b_1 &= 2 & a_1^2 &= 1 & b_1^2 &= 4; \\
a_1 &= 1.4 & b_2 &= 1.5 & a_2^2 &= 1.96 & b_2^2 &= 2.25; \\
a_2 &= 1.41 & b_3 &= 1.42 & a_3^2 &= 1.9881 & b_3^2 &= 2.0164; \\
a_3 &= 1.414 & b_4 &= 1.415 & a_4^2 &= 1.99396 & b_4^2 &= 2.002225; \\
a_4 &= 1.4142 & b_5 &= 1.4143 & a_5^2 &= 1.99996164 & b_5^2 &= 2.00024449; \\
\cdots & & \cdots && \cdots && \cdots
\end{align*}

首先说明几个概念
\begin{definition}{上界、下界、上确界、下确界}
    设\(\leq\)为定义在集合\(U\)上的序关系,\(S\)是\(U\)的子集,对于\(x \in U\),若\(\forall y\in S: y\leq x\),则称\(x\)是\(S\)的\uline{上界},\(S\)的最小上界称为\uline{上确界}(supremum),记作\(\sup(S)\);
    \par
    同理,若对于\(x \in U\),若\(\forall y\in S: x\leq y\),则称\(x\)是\(S\)的\uline{下界},\(S\)的最大下界称为\uline{下确界}(infimum),记作\(\inf(S)\).
\end{definition}

回到构造好的两个数列,随着\(n\)的增加,\(a_n^2\)和\(b_n^2\)都越来越接近\(2\),而且精确程度可以任意高,但\(a_n^2\)总是不足的,\(b_n^2\)总是过剩的,因此数列\(\{b_n\}\)中的每一项都可以作为数列\(\{a_n\}\)的上界. 可麻烦就在于\(\{b_n\}\)没有最小值(可以用数论的方法证明\(\sqrt{2}\)不是有理数),即\(\{a_n\}\)的上确界不存在,或者说\(\trm{sup}\{a_n\}\not \in \mathbb{Q}\),有理数集被\(\trm{sup},\trm{inf}\)运算捅出了漏洞.

以上构造的两个数列有非常明显的特征,它们不一定需要后一项与前一项之差越来越小,但必须控制在一个范围内,更近一步说,任意两项之差都必须控制在一个范围内,这一类数列就是\uline{柯西数列}.
\begin{definition}{柯西数列}
    在度量空间\((S,d)\)中定义序列\(\{a_n\}_{n=0}^{\infty}\),若对于任意的\(\varepsilon \in \mathbb{Q}^+\),都存在正整数\(N\),使得任意\(m,n>N\),都有\(d(a_{m},a_{n}) < \varepsilon\),则称该序列为\uline{柯西序列}(Cauchy sequence).
    用有理数域上的语言来叙述:
        对于有理数列\(\{a_n\}_{n=0}^{\infty}\),若对于任意的\(\varepsilon \mathbb{Q}^+\),都存在正整数\(N\),使得任意\(m,n>N\),都有\(|a_{m}-a_{n}| < \varepsilon\),则称数列\(\{a_n\}\)为有理数域上的柯西数列.
\end{definition}
需要强调,以上构造的两个数列定义在有理数的范围内,现在也可以说,有理数集被柯西数列捅出了漏洞. 构造实数的关键一步在于把这些漏洞补上,使得无论怎么构造柯西数列,都不会收敛到集合外面,即所谓“柯西完备性”.

\begin{definition}{完备性}
    若定义了序关系的集合\(U\)的每个有界子集都有上确界,则称该序关系是\uline{序完备的}(order-complete);
    \par
    若定义了序关系的集合\(U\)上的每个柯西数列都收敛于\(U\)内,则称该序关系是\uline{柯西完备的}(Cauchy-complete).
\end{definition}

在将要构造的实数域内,序完备和柯西完备是等价的.

尽管\(\{a_n\}\)和\(\{b_n\}\)不收敛于有理数,但随着\(n\)的增大,它们的差距越来越小,用有理数的\(\varepsilon-N\)语言来说:\(\forall \varepsilon > 0\),\(\exists N\in \mathbb{N}\),当\(n > N\)时,有\(|a_n-b_n|<\varepsilon\). 直观理解,尽管它们不收敛到有理数,但似乎收敛于同一个东西。我们称满足这个关系的两个柯西数列\(\{a_n\}\)和\(\{b_n\}\)为\uline{等价的},并记做\(\{a_n\} \equiv \{b_n\}\),然后将和\(\{a_n\}\)等价的全体柯西数列的集合记作\([\{a_n\}]_{\equiv}\),称为\uline{柯西等价类},并把全体有理数柯西数列的集合记作\(\mathcal{C}\).

每一个柯西等价类中的柯西数列都收敛于同一个目标,这个“目标”可能是有理数,也可能不是,但至少可以知道,这些“目标”组成的集合可以使得有理数的大小关系扩充为一个完备的序关系,这样的“目标”,在直觉上可以理解为实数,但严谨起见,我们不能把一个模糊而抽象的“目标”当做被定义的对象,所以干脆用等价类本身算了,因此其大小关系和相等关系也需要重新定义.
\begin{definition}{实数}
    有理数集的柯西等价类称为\uline{实数}(real number),全体实数的集合记为\(\mathbb{R}\),即
    \[\mathbb{R} = \{[\{a_n\}]_{\equiv} | \{a_n\}\in \mathcal{C}\}\]
    \newline
    对于两个实数\(a,b\),从其中分别任意抽出一个柯西数列\(\{a_n\}\)和\(\{b_n\}\),若存在\(N\),使得对于任意\(n>N\),都有\(a_n \geq b_n\),则记\(a \geq b\).
    \newline
    对于两个实数\(a,b\),从其中分别任意抽出一个柯西数列\(\{a_n\}\)和\(\{b_n\}\),若\(\{a_n\} \equiv \{b_n\}\),则记\(a=b\).
\end{definition}

\subsection{实数的十进制表示}

接下来要解决实数的十进制表示问题. 用十进制表示实数是一个很显然的过程,但完整地解释起来并不是那么简单,尤其是出现省略号的时候. 为什么\(12.333\cdots\)是实数的合法表达,而\(\cdots333.21\)却不是?为什么\(00123\)和\(123\)表示的是同一个实数,而\(123\)和\(12300\)却不是?我们是否能跨越无穷多位小数,更改“最后”几位数,捏造出\(12.333\cdots334\)这样的数出来,并认为它有别于\(12.333\cdots\)?以及贴吧的经验密码,为什么\(0.999\cdots\)和\(1\)是同一个实数?接下来依次定义整数的十进制表示、有限小数的十进制表示和无限小数的十进制表示来解决这个问题.

\begin{definition}{位}
    实数十进制表示的\uline{位}(digit)是一个自然数,而且只能取\(0,1,2,3,4,5,6,7,8,9\)之一,再定义自然数的符号\(\mbox{拾}\)表示\(9\)的后继数.
\end{definition}

这里使用符号“拾”而不是\(10\)是因为现在尚未定义出十进制表示法,使用\(10\)会导致逻辑循环. 现在使用一个有限长的由位组成的数列来表示整数部分.

\begin{definition}{整数的十进制表示}
    对于一个由位组成的数列\(\{a_i\}_{i=0}^n\),定义字符串
    \[\pm a_n a_{n-1} a_{n-2} \cdots a_1 a_0 := \pm \sum_{i=0}^{n} a_i \times \mbox{拾}^i\]
    称为整数的十进制表示;
    \newline
    若\(a_n \neq 0\),则该字符串又称为整数的约化十进制表示.
\end{definition}

整数存在唯一的约化十进制表示,这点可用带余除法证明;而十进制表示可以有任意多个,因为如果\(a_n a_{n-1}\cdots a_1 a_0\)是整数的十进制表示,那么\(0a_n a_{n-1}\cdots a_1 a_0\)也是. 因此现在可以说\(00123\)和\(123\)表示同一个整数,而\(123\)和\(12300\)表示两个不同整数. 此外\(10=1 \times \mbox{拾}^1+0=\mbox{拾}\),因此现在可以用\(10\)来代替字符“拾”了.

\begin{definition}{有限小数的十进制表示}
    对于一个由位组成的数列\(\{a_i\}_{i=-m}^n\),定义字符串
    \[\pm a_n a_{n-1} a_{n-2} \cdots a_1 a_0.a_{-1} a_{-2} \cdots a_{-m}:= \pm \sum_{i=-m}^{n} a_i \times 10^i\]
    称为有限小数的十进制表示;
    \newline
    若\(a_n \neq 0\)且\(a_{-m} \neq 0\),则该字符串又称为有限小数的约化十进制表示.
\end{definition}

同样地,有限小数也存在唯一的约化十进制表示,但有任意多个十进制表示.

从有限到无限的拓展必须十分小心,无穷多个数的和必须要另外定义\footnote{对于数列\(\{a_n\}_{n=0}^{\infty}\),若对于任意\(\varepsilon > 0\),都存在正整数\(N\),使得当\(n>N\)时,有\(\displaystyle{\left|\sum_{i=0}^{n}a_i-L\right| < \varepsilon}\),则称该级数是收敛的,且收敛于\(L\),写作\[\sum_{n=0}^{\infty}a_n=L\]此为级数的\uline{柯西和},当无穷出现在求和号下方时定义类似.}.

\begin{definition}{无限小数的十进制表示}
    对于一个由位组成的数列\(\{a_i\}_{i=-\infty}^n\),定义字符串
    \[\pm a_n a_{n-1} a_{n-2} \cdots a_1 a_0.a_{-1} a_{-2} \cdots := \pm \sum_{i=-m}^{n} a_i \times 10^i\]
    称为无限小数的十进制表示;
    \newline
    若\(a_n \neq 0\),则该字符串又称为无限小数的约化十进制表示.
\end{definition}

无限小数的约化十进制表示并不是唯一的,但每一个实数都至少有一个约化十进制表示.

\begin{proposition}{百度贴吧经验密码}
    在标准实数系下,
    \[0.999\cdots = 1\]
\end{proposition}
\textit{
    证明:这里需要说明,\(0.999\cdots\)中省略号省略了无穷多位\(9\). 因此根据定义,
    \[0.999\cdots := \sum_{n=-\infty}^{-1} 9 \times 10^{n} = \sum_{n=1}^{\infty} 9 \times 10^{-n}\]
    任意给定\(\varepsilon > 0\),取\(N=-\lceil\lg \varepsilon\rceil\),当\(n>N\)时,根据等比数列求和公式
    \[\left| 1-\left(\sum_{i=1}^{n} 9 \times 10^{-i}\right) \right| = \left| 1-\frac{9\times 10^{-1}(1-10^{-i})}{1-10^{-1}} \right| =10^{-n}<10^{-N}<\varepsilon\]
    即该级数的柯西和收敛于\(1\),因此\(0.999\cdots = 1\).
}


\subsection{可数与不可数}

以前提到过“可数/可列”与“不可数/不可列”的概念,比如概率的可列可加性,却没有详细解释“可列”是什么意思,这里给出详细的讨论.

首先区分一下潜无穷和实无穷. 以前我们说的“数列/函数的极限是无穷”中的“无穷”是潜在的无穷,通俗地说,数列/函数的取值随着某个条件“逐渐被满足”就可以超越任意有限的对象. 把无穷当做变化着成长着被不断产生出来的东西来解释,就是所谓的“潜在的无穷”,简称“\uline{潜无穷}”(potential infinity). 而\uline{实无穷}(actual infinity)是一种已经构造出来的、完成了的无穷实体,不需要用一个过程来解释,比如自然数集\(\mathbb{N}\)、整数集\(\mathbb{Z}\)、复数集\(\mathbb{C}\)都是实无穷的典型代表.

无穷是可以比较大小的,潜无穷大小的比较往往看的是“过程的速度”,高阶无穷小就是基于此提出来的. 而实无穷(含有无穷元素的集合)大小的比较则是比较谁的元素更多,这就涉及到“多”的定义. 一般比较多少有两种方法,一种是直接数个数;另一种是一一对应,谁剩谁就多. 有穷集合两种都可以用,无穷集合只能用后者.

严格来说,这种一一对应就是所谓“双射”,现在给出形式定义\footnote{card是cardinal(基数)的简写,该定义有助于理解\(\trm{card}(A)\)想表达的意思,但实际上这并不是严格定义,而有穷集合的基数等于集合元素的个数也依赖于冯·诺依曼的序数定义. 在承认选择公理的基础上,\(\trm{card}(A)\)定义为能与该集合建立双射的最小序数.}
\begin{definition}{集合的势(集合的基数)}
    如果能建立单射\(f:A \to B\),则记\(\trm{card}(A) \leq \trm{card}(B)\);
    \newline
    如果该单射能射满\(B\),就称\(A\)和\(B\)是\uline{等势的}(equinumerous),记作\(\trm{card}(A)=\trm{card}(B)\);
    \newline
    如果该单射不能射满\(B\),那么说明\(B\)的元素“更多”,记作\(\trm{card}(A)<\trm{card}(B)\);
    \newline
    对于有穷集合\(A\),\(\trm{card}(A)\)表示其元素的个数.
\end{definition}

在承认选择公理的条件下\footnote{不承认选择公理的话就会出现一些非常混乱的集合,比如六种戴德金无穷,测度论将会遭受毁灭性的打击,数学分析也将难以进行.},自然数集\(\mathbb{N}\)是最小的无穷,也就是说如果\(X\)为无穷集合,就必有\(\trm{card}(X)\geq\trm{card}(\mathbb{N})\). 如果无穷集合元素的个数与自然数“一样多”,就会存在一个序列\(\{a_n\}_{n\in\mathbb{N}}\)能把\(X\)中的元素一一枚举出来,这便是所谓“可数”或者“可列”.

\begin{definition}{可数/可列与不可数/不可列}
    若存在自然数\(n\),使得\(\trm{card}(X)=n\),则称\(X\)是有穷的(finite).
    \par
    若\(\trm{card}(X)=\trm{card}(\mathbb{N})\),则称\(X\)是\uline{可数的}(countable)或\uline{可列的}(enumerable). 
    \par
    若\(\trm{card}(X)>\trm{card}(\mathbb{N})\),那么\(X\)就是\uline{不可数的}(uncountable)或\uline{不可列的}(denumerable).
    \par
    有限集和可数无穷集合称\uline{至多可数集}.
\end{definition}

关于基数的定义,有以下几个命题符合我们心目中关于“集合元素的个数”的直觉:
\begin{itemize}
    \item [(1)] 若\(\trm{card}(A)=\trm{card}(B), \trm{card}(B)=\trm{card}(C)\),则\(\trm{card}(A)=\trm{card}(C)\).
    \item [(2)] 若\(A \subseteq B\),则\(\trm{card}(A) \leq \trm{card}(B)\).
    \item [(3)] 若\(A,B\)都是有穷集合,则\(\trm{card}(A \times B) = \trm{card}(A) \cdot \trm{card}(B)\).
    \item [(4)] (依赖于选择公理)任意不可数无穷集合都存在可数无穷子集.
    \item [(5)] \uline{康托尔-伯恩斯坦-施罗德定理}(Cantor-Bernstein-Schröder theorem):
    \newline
    若\(\trm{card}(X) \leq \trm{card}(Y)\)且\(\trm{card}(X) \geq \trm{card}(Y)\),则\(\trm{card}(X) = \trm{card}(Y)\),即基数的大小关系具有反自反性.
\end{itemize}

在有限集合上,“数个数”和“配对”的方法得出的结果都是符合直觉的,然而涉及到了无穷,情况就有些违反直觉.
\begin{itemize}
    \item [(1)] 对于无穷集合\(A\),有\(\trm{card}(A \times A) = \trm{card}(A)\).
    \newline
    \textit{
        证明:完整的证明需要用到超限归纳法,所以这里来证个弱化的版本:若\(A\)为可数无穷,则\(\trm{card}(A \times A) = \trm{card}(A)\).
        \newline
        \(f(x) = \langle x, x \rangle: A \to (A \times A)\)明显是单射,所以\(\trm{card}(A) \leq \trm{card}(A \times A)\). 接下来证明\(\trm{card}(A) \geq \trm{card}(A \times A)\). 由于\(A\)可数,所以它的元素可以被序列\(\{a_n\}_{n \in \mathbb{N}}\)枚举,所以\(A \times A = \{\langle a_i, a_j \rangle\}_{i,j \in \mathbb{N}}\).这时候定义映射\(f(a_i,a_j) = a_{2^i 3^j}\),根据算数基本定理\footnote{算数基本定理(fundamental theorem of arithmetic):每个大于\(1\)的自然数可以唯一地分解为有限个素数的乘积.},\(2^i3^j\)对于不同的\(i,j\)值得到不同的自然数,因此\(f\)是一个单射,即\(\trm{card}(A) \geq \trm{card}(A \times A)\),最终得到\(\trm{card}(A) = \trm{card}(A \times A)\).
    }
    根据这个结果,\(\trm{card}(\mathbb{N})=\trm{card}(\mathbb{Q})\),而\(\mathbb{Z} \subset \mathbb{Q}\),所以\(\trm{card}(\mathbb{N})=\trm{card}(\mathbb{Z})=\trm{card}(\mathbb{Q})\). 此外,\(n\)维空间中的有理点、奇数集、偶数集、素数集都可数;

    \item [(2)] \uline{康托尔定理}(Cantor's theorem):
    \newline
    定义\(\mathcal{P}(A):=\{x | x \subseteq A\}\),称为\(A\)的\uline{幂集}(power set),则\(\trm{card}(\mathcal{P}(A)) > \trm{card}(A)\).
    \newline
    \textit{
        证明:\(\trm{card}(\mathcal{P}(A)) \geq \trm{card}(A)\)是显然的,因为\(f(x)=\{x\}: A \to \mathcal{P}(A)\)是单射,重点在于证明\(\trm{card}(\mathcal{P}(A)) \neq \trm{card}(A)\). 反证法,假设存在双射\(g:A \to \mathcal{P}(A)\),构造\(A\)的子集\(B=\{x \in A | x \not \in g(x)\}\),\(B \in \mathcal{P}(A)\),所以存在\(x \in A\)使得\(f(x)=B\),然后矛盾就出来了:
        \[x \in B \Longleftrightarrow x \not \in g(x) \Longleftrightarrow x \not \in B\]
        因此该双射是不存在的,\(\trm{card}(\mathcal{P}(A)) > \trm{card}(A)\).
    }

    \item [(3)] 记\(X^Y=\{f|f:Y \to X\}\),则\(\trm{card}\left(\{0,1\}^A\right) = \trm{card}(\mathcal{P}(A))\).
    \newline
    \textit{
        证明:利用示性函数. 定义\(f(S) = \chi_S:\mathcal{P}(A) \to \{f | f: A \to \{0,1\}\}\). 易验证\(f\)是双射. 因为从\(\mathcal{P}(A)\)中挑出任意子集,得到的示性函数\(\chi_S\)也不同.
    }

    \item [(4)] \(\trm{card}(\mathbb{R}) = \trm{card}(\mathcal{P}(\mathbb{N})) \)
    \newline
    \textit{
        我们不直接证明原命题,而证\(\trm{card}(\mathbb{R})=\trm{card}\left(\{0,1\}^\mathbb{N}\right)\),原命题即可根据上个性质得证.
        \newline
        根据双射\(f(x)=\tan\left(\frac{\pi}{2}x-\frac{1}{2}\right)\),易知\(\trm{card}(0,1)=\trm{card}(\mathbb{R})\),而\((0,1)\)中的每个实数都可以用二进制小数\(0.a_1a_2a_3\cdots\quad(a_i=0,1)\)来表示,令\(S=\{i|a_i=1\}\),则\(\chi_S(n)\)便是一个根据该小数构造出来的示性函数. 和上一个性质证明同样的思路,易得\(\trm{card}(\mathbb{R})=\trm{card}\left(\{0,1\}^\mathbb{N}\right)\).
    }
    根据这个结果,\(n\)维空间中的点、连续几何曲线上的点、无理数集、有理柯西数列的集合都不可数,并与\(\mathbb{R}\)等势.
    
\end{itemize}

实数不可数最初由康托尔(Georg Cantor)通过对角线证法得出,这个证明发表的那一天标志着集合论的诞生,尽管以今天的视角来看,当时的证明有些瑕疵(无限小数\(p\)进制表示的不唯一性),但这些瑕疵都很好解决,比如约定禁止\(9\)循环出现,或者用连分数展开的线性表示式来代替无限小数等.


\subsection{实数完备性定理}

可以用以下几个命题描述实数完备性:
\begin{theorem}{确界存在性/戴德金完备性}
    \begin{itemize}
        \item [\(\bullet\)] \uline{戴德金完备性}(Dedekind completeness):
        \par 对有序域\(S,\leq\)作划分\(\mathfrak{P} = \{A,B\}\),则\((\forall a \in A)(\forall b \in B):[a \leq b \Longrightarrow \exists c \in S:(a \leq c \leq b)]\)
        \item [\(\bullet\)] 在实数域中叙述为:
        \par 对于\(\mathbb{R}\)的任意非空子集,只要有上界,就有上确界;
        \par 把\(\mathbb{R}\)的元素分到两个集合\(A,B\)中,并使\(A\)中的任意数小于\(B\)中的任意数,那么一定存在实数\(c\),对于\(A\)中的元素\(a\)和\(b\)中的元素\(b\),都满足\(a\leq c\leq b\).
    \end{itemize}
\end{theorem}

\textit{戴德金完备性通俗解释就是用一把刀把数轴砍成两段,无论刀刃有多薄,都会砍中一个实数}.

\begin{theorem}{柯西完备性}
    \begin{itemize}
        \item [\(\bullet\)]\uline{柯西完备性}(Cauchy completeness):
        \par 对于距离空间\((S,d)\)中的任意柯西序列\(\{a_n\}_{n=0}^{\infty}\),都存在\(a \in S\),对于任意\(\varepsilon>0\),存在自然数\(N\),使得\(n>N\)时,\(d(a,a_n) < \varepsilon\).
        \item [\(\bullet\)] 在实数域中叙述为:
        \par \(\mathbb{R}\)上的数列\(\{a_n\}_{n=0}^{\infty}\)收敛的充要条件是对于任意\(\varepsilon>0\),存在\(N\)使得\(m,n>N\)时,\(|a_n-a_m| < \varepsilon\).
        \item [\(\bullet\)] 推论:\uline{单调有界定理/分析学基本定理}
        \par 单调递增且有上界的数列必然收敛.
    \end{itemize}
\end{theorem}

\begin{theorem}{闭区间套定理}
    \begin{itemize}
        \item [\(\bullet\)] \uline{康托尔交集定理}(Cantor intersection theorem):
        \par 在拓扑空间中定义非空有界闭集合列\(\{S_n\}_{n=0}^{\infty}\),并使\(S_n \supseteq S_{n+1}\),则\(\bigcap_{n=0}^{\infty}S_n \neq \emptyset\);当且仅当该拓扑空间为柯西完备的距离空间时,存在唯一的元素\(a\),使得\(\bigcap_{n=0}^{\infty}S_n =\{a\}\).
        \item [\(\bullet\)] 在实数域中叙述为:
        \par 对于一系列闭区间\(\{[a_n,b_n]\}_{n=0}^{\infty}\),若\([a_n,b_n]\supseteq [a_{n+1},b_{n+1}]\), 而且\(\lim \limits_{\substack{n \to \infty}}a_n-b_n=0\),则存在唯一实数\(L\),使得\(\lim \limits_{\substack{n\to\infty}}a_n=\lim \limits_{\substack{n\to\infty}}b_n=L\).
    \end{itemize}   
\end{theorem}

\textit{这一系列前一个包住后一个的闭区间就称为\uline{闭区间套},这个定理说的是,即使闭区间的长度趋于\(0\),依然存在一个实数,属于这一系列所有的闭区间}.

\begin{theorem}{有界紧性质/有限覆盖定理}
    \begin{itemize}
        \item [\(\bullet\)] \uline{海涅-波莱尔定理}(Heine-Borel theorem):
        \par 欧氏空间\(\mathbb{R}^n\)中有界闭集与紧集等价.
        \item [\(\bullet\)] 在实数域中叙述为:
        \par 如果有一系列开区间覆盖了一个闭区间,那么可以从这个系列中选出有限个开区间覆盖这个闭区间.
    \end{itemize}
\end{theorem}

\begin{theorem}{列紧性质/聚点定理}
    \begin{itemize}
        \item [\(\bullet\)] \uline{波尔查诺-魏尔斯特拉斯定理}(Bolzano-Weierstrass theorem):
        \par 欧氏空间\(\mathbb{R}^n\)中每一个有界序列必有收敛子列.
        \item [\(\bullet\)] 在实数域中叙述为:
        \par \(\mathbb{R}\)有界数列必有收敛子列;
        \par \(\mathbb{R}\)上的无穷有界集含有至少一个聚点.
    \end{itemize}
\end{theorem}

\begin{theorem}{介值定理(intermediate value theorem)}
    \(\mathbb{R}\)上定义的连续函数,若能取到两个不同的值\(a,b\),则可以取到闭区间\([a,b]\)内的所有实数.
\end{theorem}

\textit{
    函数的连续性是拓扑性质,不依赖实数的完备性. 设函数\(f:X \to Y\),其中\(X,Y\)为度量空间,距离函数为\(d_X,d_Y\),若\(f\)在\(x_0 \in X\)处有定义,且对于任意\(\varepsilon > 0\),都存在\(\delta > 0\),当\(d_X(x,x_0)<\delta\)时,有\(d_Y(f(x),f(x_0))<\varepsilon\),就称函数\(f\)在\(x_0\)处连续.
}

\begin{theorem}{苏斯林性质(Suslin property)}
    若有一系列开区间两两不相交,则这些开区间至多是可数的.
\end{theorem}

\textit{这条性质实际上是“有理数在实数集中稠密”的另一种表述,由于每个开区间至少要包含一个有理数,所以这一系列开区间不能“多于”有理数.}

\subsection{关于实数认识的局限性}

实数集是不可数无穷的,它是比自然数的可数无穷更高阶的无穷,人们对实数的认识目前还有很多局限性,关于实数性质仍有很多奇妙的地方.

之前的讨论确认了\(\trm{card}(\mathbb{N})<\trm{card}(\mathbb{R})\),于是自然有这样的问题:是否存在一个集合\(S\),使得 \(\trm{card}(\mathbb{N}) < \trm{card}(S) < \trm{card}(\mathbb{R})\)呢?连续统假设给出了否定回答.

\begin{proposition}{连续统假设(continuum hypothesis)}
    不存在集合\(S\),使得\(\trm{card}(\mathbb{N})<\trm{card}(S)<\trm{card}(\mathbb{R})\).
\end{proposition}

连续统假设尚未被证明或证伪,而且研究还发现,这个问题实在太过于特殊,现有的公理体系以及将来进行的一切构造性拓展都不足以证明或证伪它,除非承认一些纲领性的命题;甚至在某些哲学流派看来,连续统假设甚至都不是一个定义良好的问题. 目前学界倾向于否定它.

实数集以及定义在其上的四则运算和大小关系构成一个完备的有序域,其重要特征是(1)无最大或最小的元素;(2)稠密性;(3)完备性;(4)苏斯林性质. 符合这四个条件的完备有序域不止有\(\mathbb{R}\),还可以构造出很多. 问题来了,这四条性质是否足够刻画实数?对于另一个符合这四个特征的完备有序域\(S\),是否一定存在一个双射\(f:S\to \mathbb{R}\),使得若\(a\leq b\),则\(f(a)\leq f(b)\)呢?苏斯林假设给出了肯定回答.
\begin{proposition}{苏斯林假设(Suslin hypothesis)}
    满足这四个特征的完备有序域一定存在到实数集上的保序映射.
\end{proposition}

苏斯林假设也尚未被证明或证伪,目前的研究结果是,如果把连续统假设当成公理加入现有的公理体系,再放入一些构造性的命题,就可以导出苏斯林假设. 如果苏斯林假设成立,就说明不必借助代数结构刻画实数.

\vspace{1cm}

无论是算法还是普通的描述性语句,它们都是有限长的符号序列,而前者是后者的一种特殊情况. 所有算法组成的集合和所有精确描述一个实数的语句组成的集合都是可数无穷多个的,面对不可数无穷多个的实数,必然有一些是无法计算和无法描述的,它们是比一般的无理数更加“无理”的数. 从有理数扩张到实数可以依次经过以下这几个具有里程碑意义的集合:
\begin{itemize}
    \item [(1)]
    代数数(algebric number). 代数数是整系数高次多项式方程的实根,有限次使用四则运算和乘方开方得到的数都是代数数,比如常见的\(\sqrt2\)、 黄金分割比\(\dfrac{\sqrt{5}-1}{2}\)等都是代数数. 全体代数数集合的势为\(\trm{card}(\mathbb{N})\),代数数在实数内的补集称为超越数(transcendental number). 我们接触到的具体的数大多都是代数数,不过超越数也很容易借助\(e\)得到:
    \begin{theorem}{林德曼-魏尔斯特拉斯定理(Lindermann-Weierstrass theorem)}
        对于非零代数数\(a_1, a_2, \cdots, a_n\)和代数数\(b_1, b_2, \cdots, b_n\),均有
        \[a_1e^{b_1}+a_2e^{b_2}+\cdots+a_ne^{b_n} \neq 0\]
    \end{theorem}
    以下定理则更进一步,说明超越数可以用两个代数数生成.
    \begin{theorem}{格尔丰德-施耐德定理(Gelfond–Schneider theorem)}
        若\(a\)是不为\(0,1\)的代数数,\(b\)是无理数,则\(a^b\)是超越数.
    \end{theorem}
    这两个定理直接说明了对于\(0\)以外的代数数\(x\),\(e^x\)均为超越数,它的反函数\(\ln(x)\)也只经过\((1,0)\)这个唯一的横纵坐标都是代数数的点,套入欧拉公式,可以得出除了几个特殊的点以外,三角函数和双曲函数也不经过横纵坐标都是代数数的点,比如\(\sin(1)\)就是一个超越数.
\end{itemize}
\begin{itemize}
    \item[(2)]
    可计算数(computable number). 尽管超越数不能通过解整系数的高次多项式方程得到,比如\(\pi, e\)等,但是它们也许存在解析形式的表达式,或者有一个算法可以把它们计算到任意精确的程度,通俗的说就是“知道它怎么算”. 全体可计算数的势为\(\trm{card}(\mathbb{N})\). 不可计算意味着无从得知某些量精确的值,但是可以证明这些量是一个定值,比如蔡廷常数(Chaitin's constant)\(\Omega\),不严谨地定义为随机一段可以运行的程序最终能停下来的概率,根据概率的性质可以知道\(0<\Omega<1\),但没有更准确的范围了.
\end{itemize}
\begin{itemize}
    \item[(3)]
    可定义数(definable number). 对于这类数,只能通过非构造式的方法意识到它们的存在,但它们中的任意一个元素都无法被定义出来,因为一旦做到了,就不是不可定义数了. 证明的思路很简单\footnote{这种把字符串重新编码为一个自然数的证明思路称为哥德尔编码(Godel number)方法},任意一个“定义”都是一串长度有限的字符串,而字符串中的每个不可分割的字符可能的选取都是有限的(我们一般把包含所有不可分割的字符的集合称为字母表或者字典),把字典中的字符编号,并用编号代替字符本身,那么每一个字符串就可以写成一个长长的自然数,这样就建立了所有“定义”的集合到自然数集的单射,实数集远远要大于自然数集,因此有些实数是不可被定义的.
\end{itemize}

假设有一个集合\(A\),对于一个实数,如果我们能证明它是无理数,就把它加入集合\(A\),由于“证明”的过程也是一个长度有限的字符序列,所以可以得到\(A\cup\mathbb{Q}\neq \mathbb{R}\),并且\(\trm{card}(A\cup\mathbb{Q})<\trm{card}(\mathbb{R})\). 若往现有的公理体系中加入更多更强的公理,就可以扩充\(A\),但并不会改变它的势,我们完全可以找到一个实数,在现有公理体系的某个模型下证明它是无理数,在另一个模型下证明它是无理数. 说到底,有理和无理的性质已经触及到了数学大厦的总体面貌,非常依赖于讨论时定下的哲学纲领,关于它的研究目前是一个较为热门的方向.

\end{document}
