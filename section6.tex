
\documentclass[main.tex]{subfiles}

\begin{document}

\section{欧拉数}

欧拉有很多的数学成果,和“欧拉数”这三个字有关的东西就有Euler numbers, Euler's number, Eulerian number, Euler's constant, Eulerian integers等,这里的欧拉数指的是第一个,和上面的伯努利数一样是一个数列\(\{E_n\}_{n=0}^{\infty}\).

欧拉数由生成函数导出:
\[ \frac{1}{\cosh(x)} = \frac{2}{e^{x}+e^{-x}} = \sum_{k=0}^{\infty}\frac{E_k}{k!}x^k\]
同伯努利数一样,欧拉数有正有负,绝对值总体趋势是增加的,但都是整数,奇数项都为0.\\
\begin{tabular}{|l|l|l|l|l|l|l|l|l|l|}
    \hline
    \(E_{0}\) & \(E_{2}\) & \(E_{4}\) & \(E_{6}\) & \(E_{8}\) & \(E_{10}\) & \(E_{12}\) & \(E_{14}\) & \(E_{16}\) & \(E_{18}\) \\
    \hline
    \(1\) & \(-1\) & \(5\) & \(-61\) & \(1385\) & \(-50521\) & \(2702765\) & \(-199360981\) & \(19391512145\) & \(−2404879675441\) \\
    \hline
\end{tabular}

有了欧拉数,我们可以给更多的函数找到通用的展开式了。
\[ \sec(x) = 1+\frac{1}{2}x^2+\frac{5}{24}x^4+\frac{61}{720}x^6+\frac{277}{8064}x^8+\frac{50521}{3628800}x^{10}+\cdots\]
首先注意到\(\displaystyle{\sec(x) = \frac{1}{\cos(x)} = \frac{2}{e^{ix}+e^{-ix}}}\)
立即代入欧拉数的导出函数中,并略去奇数项,得到
\[ \sec(x) = \sum_{k=0}^{\infty} \frac{E_k}{k!}(ix)^k = \sum_{k=0}^{\infty}\frac{(-1)^kE_{2k}}{(2k)!}x^{2k}, \quad |x|<\frac{\pi}{2}\]

与伯努利数相似,偶数项欧拉数的绝对值呈指数式增长
\[ |E_{2 n}| \approx 8 \sqrt { \frac{n}{\pi} } \left(\frac{4 n}{ \pi e}\right)^{2 n}\]
增长快慢来说,若记\(A \prec B\)表示\(A\)是\(B\)的高阶无穷大,那么对于正整数\(n\),当\(n \to +\infty\)时,有\(\psi(2n) \prec 2n \prec e^{2n} \prec |B_{2n}| \prec |E_{2n}| \prec \Gamma(2n) \prec n^{2n}\)


\end{document}
