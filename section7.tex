
\documentclass[main.tex]{subfiles}

\begin{document}

\section{欧拉积分系列}

\subsection{伽马函数}

\(\Gamma\)函数也称为第二类欧拉积分,它是阶乘的扩展,阶乘定义在自然数上,\(\Gamma\)函数则定义在复数集上,它有好几种不同的定义,但最常用的还是伯努利的积分定义:
\[ \Gamma(z) := \int_0^{+\infty} t^{z-1} e^{-t} \trm{d}t, \quad \trm{Re}(z) > 0 \]
经过简单的换元,可以得到不同的结果.
\newline
(1) 令\(t=u^2\)
\[ \Gamma(z) = 2\int_{0}^{\infty}e^{-u^2}u^{2z-1}\trm{d}u\]
(2) 令\(e^{-t}=u\)
\[ \Gamma(z) = 2\int_{0}^{1}\left(\ln\frac{1}{u}\right)^{z-1}\trm{d}u\]

定义\(\Gamma\)函数的初衷是让阶乘在实数域上有定义,但是\(\Gamma(n+1)=n!\),为了和阶乘统一起来,又定义了\(\Pi(n):=\Gamma(n+1)=n!\),但是接下来不使用它.

\(\Gamma\)函数有很多性质.
\begin{itemize}

    \item [(1)] 递推公式
    \[ {\color{blue} \Gamma(z+1) = z\Gamma(z)} \]
    \textit{
        \(\Gamma\)函数的递推公式就是根据阶乘得来的,不过假如不知道\(\Gamma\)和阶乘的关系,也可以利用分部积分法证明.
        \[ \Gamma(z+1) = \int_0^{+\infty} t^{z} e^{-t} \trm{d}t = -\int_{0}^{+\infty}t^z \trm{d}e^{-t} = \left.-t^ze^{-t}\right|_0^{+\infty}+\int_{0}^{+\infty}e^{-t}zt^{z-1} = z\Gamma(z)\]
        又因为
        \[ \Gamma(1) = \int_{0}^{\infty} e^{-t}\trm{d}t = 1\]
        所以可以确立关系\(n! = \Gamma(n+1)\),这也从一方面说明了为什么要求\(\trm{Re}(z)>0\),试想\(\Gamma(0) = (-1)\times(-2)\times(-3)\times\cdots\),这是个无穷递降乘积,不但不收敛,还会以越来越大的幅度震荡,永远也求不出它的值.
    }

    \item[(2)] 欧拉的无穷乘积定义
    \[\Gamma(z) = \frac{1}{z}\prod_{k=1}^{\infty} \frac{(1+\frac{1}{k})^z}{1+\frac{k}{z}}, \quad z \neq 0, -1, -2, \cdots\]
    \textit{
        欧拉提出无穷乘积定义是从这个极限开始的:\(\displaystyle{\lim_{n \to \infty} \frac{n!(n+1)^m}{(n+m)!}}=1\). 这个极限证明起来比较直观,把阶乘展开再化简就好了:
        \[\lim_{n \to \infty} \frac{n!(n+1)^m}{(n+m)!} = \lim_{n \to \infty} \frac{(n+1)(n+1) \cdots (n+1)}{(n+2)(n+3)\cdots(n+m)} = \lim_{n \to \infty} \frac{n+1}{n+2} \cdot \lim_{n \to \infty} \frac{n+1}{n+3} \cdots \lim_{n \to \infty} \frac{n+1}{n+m} = 1\]
        两边同时乘\(m!\),并展开,就可以得到:
        \[m! = \lim_{n \to \infty} \frac{m!n!(n+1)^m}{(n+m)!} = \lim_{n \to \infty} \frac{n!(n+1)^m}{A_{m+n}^n} = \lim_{n \to \infty} \frac{\left( \frac{2}{1}\frac{3}{2}\cdots\frac{n+1}{n}\right)^m}{(1+\frac{m}{1})(1+\frac{m}{2})\cdots(1+\frac{m}{n})}\]
        写成连乘积的形式:
        \[ m! = \lim_{n \to \infty} \frac{\prod_{k=1}^{n}\left(1+\frac{1}{k}\right)^m}{\prod_{k=1}^{n}(1+\frac{m}{k})} = \prod_{k=1}^{\infty}\frac{(1+\frac{1}{k})^m}{1+\frac{m}{k}}\]
        然后把这个极限扩展到复数域上,并根据关系式\(z!=\Gamma(z+1) = z\Gamma(z)\),就得到了
        \[\Gamma(z) = \frac{1}{z}\prod_{k=1}^{\infty} \frac{(1+\frac{1}{k})^z}{1+\frac{k}{z}}\]
    }
    在欧拉无穷乘积的基础上,还可以再改写式子:
    \begin{align*}
        \Gamma(z) &= \lim_{n \to \infty}\frac{1}{z}\prod_{k=1}^{n}\left(1+\frac{1}{n}\right)^z\left(1+\frac{z}{n}\right)^{-1} = \lim_{n \to \infty} \frac{1}{z} \frac{(\frac{2}{1})^z(\frac{3}{2})^z\cdots(\frac{n}{n-1})^z(\frac{n+1}{n})^z}{(\frac{1+z}{1})(\frac{2+z}{2})\cdots(\frac{n+z}{n})} \\
        &= \lim_{n \to \infty}\frac{n!n^z}{z(z+1)\cdots(z+n)}\left(\frac{n+1}{n}\right)^z = \lim_{n \to \infty} \frac{n!n^z}{z(z+1)\cdots(z+n)} \\
        &= \lim_{n \to \infty} \frac{n!n^z}{(z)_{n+1}}
    \end{align*}
     
    \item[(3)] 魏尔斯特拉斯的无穷乘积定义
    \[\Gamma(z) = \frac{e^{-\gamma z}}{z}\prod_{n=0}^{\infty}\left( 1+\frac{z}{n} \right)^{-1} e^{\frac{z}{n}}, \quad z \neq 0, -1, -2, \cdots\]
    \textit{接上面,改写的欧拉无穷乘积可以再次改写:}
    \begin{align*}
        \Gamma(z) &= \lim_{n \to \infty} \frac{{\color{red}n!}{\color{blue}n^z}}{{\color{red}z(z+1)\cdots(z+n)}} = \lim_{n \to \infty} {\color{red}\frac{1}{z}\prod_{k=1}^{n}\left(\frac{k}{z+k}\right)}{\color{blue}\exp\left[z\ln(n)\right]} \\
        &= \lim_{n \to \infty} {\color{red}\frac{1}{z}\prod_{k=1}^{n}\left(1+\frac{z}{k}\right)^{-1}}{\color{blue}\exp\left[z\left(\ln(n) - \sum_{k=1}^{n}\frac{1}{k}\right)\right]\prod_{k=1}^{n}e^{z/k}} \\
        &= \frac{{\color{blue}e^{-\gamma z}}}{{\color{red}z}}\prod_{k=1}^{\infty}{\color{red}\left(1+\frac{z}{k}\right)^{-1}}{\color{blue}e^{z/k}}
    \end{align*}
    其中
    \[\gamma = \lim_{n \to \infty} \left(\sum_{k=1}^{n}\frac{1}{k} - \ln(n)\right) = 0.577215664\cdots \]
    称为欧拉-马歇罗尼常数(Euler-Mascheroni constant),接下来有一节会专门提到它.
    \item[(4)] \uline{余元公式}(reflection formula) 
    \[\Gamma(1+z)\Gamma(1-z) = \frac{\pi z}{\sin(\pi z)}, \quad z \not\in \mathbb{Z}\]
    \textit{
        证明:利用魏尔斯特拉斯无穷乘积
        \begin{align*}
            \Gamma(z)\Gamma(-z) &= \frac{e^{\gamma z}}{z}\prod_{n=1}^{\infty}\left(1+\frac{z}{n}\right)^{-1}e^{z/n}\frac{e^{-\gamma z}}{-z}\prod_{n=1}^{\infty}\left(1+\frac{-z}{n}\right)^{-1}e^{-z/n} \\
            &= -\frac{1}{z^2}\prod_{n=1}^{\infty}\left[\left(1+\frac{z}{n}\right)\left(1+\frac{z}{n}\right)\right]^{-1} = -\frac{1}{z^2}\prod_{n=1}^{\infty}\left(1-\frac{z^2}{n^2}\right)^{-1}
        \end{align*}
        之前讨论零点分解定理时,得到了这个式子:
        \[\frac{\sin(\pi z)}{\pi z} = \prod_{n=1}^{\infty}\left(1-\frac{z^2}{n^2}\right)\]
        所以有
        \[\Gamma(z)\Gamma(-z) = -\frac{\pi}{z\sin(\pi z)}\]
        考虑到\(\Gamma(1+z)=z\Gamma(z), \Gamma(1-z) = -z\Gamma(-z)\),所以有
        \[\Gamma(1+z)\Gamma(1-z) = \frac{\pi z}{\sin(\pi z)}\]
    }
    \item[(5)] 勒让德倍加公式(duplication formula)
    \[ \Gamma(2z) = \frac{\Gamma(z)\Gamma(z+\frac{1}{2})}{2^{1-2z}\sqrt{\pi}} \]
    \textit{
        证明需要用到接下来介绍的\(B\)函数.
        \[B(a,a) := \int_{0}^{1}x^{a-1}(1-x)^{a-1} \trm{d}x = \int_0^1 (x-x^2)^{a-1} \trm{d}x = \int_0^1 \left[\frac{1}{4} - \left(x-\frac{1}{2}\right)^2\right]^{a-1}\trm{d}x \]
        由于被积函数关于\(x=\dfrac{1}{2}\)对称,所以
        \[B(a,a)=2\int_0^{\frac{1}{2}} \left[\frac{1}{4} - \left(x-\frac{1}{2}\right)^2\right]^{a-1}\trm{d}x\]
        令\(\displaystyle{\left(x-\frac{1}{2}\right) = \frac{1}{4}t}\),则\(\displaystyle{\trm{d}x = -\frac{1}{4\sqrt{t}}\trm{d}t}\),积分上下限也由\(0\sim 1/2\)改成了\(1 \sim 0\),即
        \[B(a,a) = 2\int_0^1 \left(\frac{1}{4}-\frac{1}{4}t\right)^{a-1}\frac{1}{4}t^{-1/2}\trm{d}t = \frac{1}{2^{2a-1}}B\left(\frac{1}{2},a\right)\]
        根据关系\(\displaystyle{B(a,b) = \frac{\Gamma(a)\Gamma(b)}{\Gamma(a+b)}}\)以及\(\displaystyle{\Gamma\left(\frac{1}{2}\right) = \sqrt{\pi}}\),在上式两端代入,得到
        \[\frac{\Gamma(a)\Gamma(a)}{\Gamma(2a)} = \frac{\sqrt{\pi}}{2^{2a-1}}\frac{\Gamma(a)}{\Gamma\left(a+\frac{1}{2}\right)}\]
        整理一下即可.
    }
    \item[(5)] 由倍加公式可以推知,当自变量为某些复数时,有如下关系:
    \begin{align*}
        |\Gamma(n+bi)|^2 &= \frac{b\pi}{\sinh(b\pi)}\prod_{k=1}^{n-1} (k^2+b^2), \quad n \in \mathbb{N}_+ \\
        |\Gamma(-n+bi)|^2 &= \frac{\pi}{b\sinh(b\pi)}\prod_{k=1}^{n} (k^2+b^2)^{-1}, \quad n \in \mathbb{N} \\
        |\Gamma(\frac{1}{2} \pm n + bi)|^2 &= \frac{\pi}{\cosh(b\pi)}\prod_{k=1}^n \left((k-\frac{1}{2})^2+b^2\right)^{\pm 1}, \quad n \in \mathbb{N}
    \end{align*}
    代入特殊值可以得到\(\displaystyle{|\Gamma(bi)|^2 = \frac{\pi}{b \sinh (b\pi)}, \quad |\Gamma(1+bi)|^2 = \frac{b\pi}{\sinh(b\pi)}, \quad |\Gamma(\frac{1}{2}+bi)|^2 = \frac{\pi}{\cosh(b\pi)}}\)
\end{itemize}

\vspace{1cm}

% (3) 等价无穷大:
% \[ \lim_{n \to \infty} \frac{\Gamma(n+z)}{\Gamma(n+z)n^z} = 1\]
% (4) 斯特林公式(Stirling's approximation)
% \[ \lim_{x \to \infty} \frac{\Gamma(n+1)}{\sqrt{2 \pi n}(\frac{n}{e})^n} = 1\]
% 斯特林公式意味着当\(n\)特别大时,可以用\(\sqrt{2 \pi n}(\frac{n}{e})^n\)来近似代替\(n!\),两者相差\(o(1/n)(n \to +\infty)\),如果想进行更准确的估计,可以用斯特林级数:
% \[  n!\sim {\sqrt {2\pi n}}\left({\frac {n}{e}}\right)^{n}\left(1+{\frac {1}{12n}}+{\frac {1}{288n^{2}}}-{\frac {139}{51840n^{3}}}-{\frac {571}{2488320n^{4}}}+\cdots \right)\]

根据余元公式,可以由\(\displaystyle{\Gamma\left(\frac{1}{2}\right) = \frac{\pi}{\sin(\frac{1}{2}\pi)} = \pi}\)导出\(\displaystyle{\Gamma(\frac{1}{2}) = \sqrt{\pi}}\),再由递推公式得到
\[ \Gamma\left(n+\frac{1}{2}\right) = \sqrt{\pi}\cdot\frac{1}{2}\cdot\frac{3}{2}\cdot\frac{5}{2}\cdots\frac{2n-1}{2} = \frac{(2n-1)!!}{2^n} \]

\subsection{多伽马函数}

\begin{definition}{多伽马函数}
    \uline{多伽马函数}(polygamma function)是一系列函数\(\{\psi^{(n)}\}_{n=0}^{\infty}\)的统称,定义为
    \begin{align*}
        \psi^{(0)}(z) &:= \frac{\trm{d}}{\trm{d}z} \ln \Gamma(z) \\
        \psi^{(n+1)}(z) &:= \frac{\trm{d}}{\trm{d}z} \psi^{(n)}(z)\\
    \end{align*}
    其中\(\psi^{(0)}(z)\)简写为\(\psi(z)\),称为\uline{双伽马函数}(digamma function),是这一系列中最有研究价值的函数.
\end{definition}

digamma经过亿点点化简,可以得到几种等价的表达式:\\
(1) 高斯的结果:
\[ \psi(z) = \int_{0}^{\infty} \left( \frac{e^{-t}}{t} - \frac{e^{-zt}}{1-e^{-t}} \right) \trm{d}t \]
(2) 高斯结果的优化:
\[ \psi(z+1) = -\gamma+\int_0^1 \left(\frac{1-t^z}{1-t}\right) \trm{d} t \]
(3) 狄利克雷的结果:
\[ \psi(z)=\int _{0}^{\infty }\left(e^{-t}-{\frac {1}{(1+t)^{z}}}\right) {\frac {dt}{t}} \]
(4) 级数表达式:
\[ \psi (z+1) =-\gamma +\sum _{n=1}^{\infty }\left({\frac {1}{n}}-{\frac {1}{n+z}}\right), \quad z \not \in \mathbb{Z}_-\]

digamma函数具有如下性质:\\
(1) 递推公式
\[ \psi(z+1) = \psi(z)+\frac{1}{z} \]
(2) 反射性质
\[ \psi(1-z) = \psi(z)+z\cot(\pi z) \]
(3) 倍加公式
\[ \psi(2x) = \frac{1}{2}\psi(x) +\frac{1}{2}\psi\left( x+\frac{1}{2} \right) +\ln 2 \]

由于\(\psi(1) = -\gamma\),根据性质1可以得知
\[\psi(n) = -\gamma + 1 + \frac{1}{2} + \frac{1}{3} + \cdots + \frac{1}{n-1} = -\gamma+\sum_{i=1}^{n-1} \frac{1}{i} \]

\subsection{欧拉beta函数}

Beta函数又称为欧拉第一类积分,它看起来有点像二项分布的概率密度函数。
\[B(x,y):=\int_{0}^{1}t^{x-1}(1-t)^{y-1}\trm{d}t, \quad \trm{Re}(x) > 0, \trm{Re}(y)>0\]
Beta函数的表达式也可以经过简单的换元得到下面这些结果.\\
(1) 令\(t=\cos^2\theta\)
\[B(x,y) = 2\int_0^{\frac{\pi}{2}} (\sin\theta)^{2x-1}(\cos\theta)^{2y-1}\trm{d}\theta\]
(2) 令\(t=u^2\)
\[ B(x,y) = 2\int_{0}^{1}u^{2x-1}(1-u)^{y-1}\trm{d}u\]
(3) 令\(t=\dfrac{u}{a}\)
\[ B(x,y) = \frac{1}{a^{x+y-1}}\int_{0}^{a}u^{x-1}(a-u)^{y-1}\trm{d}u\]
(4) 令\(t=\dfrac{1+u}{2}\)
\[ B(x,y) = \frac{1}{2^{x+y-1}}\int_{-1}^{1}(1+u)^{x-1}(1-u)^{y-1}\trm{d}u\]
(5) 令\(t=\dfrac{u}{1+u}\)
\[ B(x,y) = \int_{0}^{\infty}\frac{u^{x-1}}{(1+u)^{x+y}}\trm{d}u\]

%    &= \sum_{k=0}^{\infty}\frac{\begin{pmatrix}k-y \\ k \end{pmatrix}}{x+k} \\
%    &= \frac{x+y}{xy}\prod_{k=1}^{\infty}\left(1+\frac{xy}{k(x+y+k)} \right)^{-1} \\
%    &= \int_0^{+\infty}\frac{t^{x-1}}{(1+t)^{x+y}}\trm{d}t, \quad \trm{Re}(x) > 0, \trm{Re}(y) > 0 \\
%    &= n\int_0^1 t^{nx-1}(1-t^n)^{y-1} \trm{d}t, \quad \trm{Re}(x) > 0, \trm{Re}(y) >0, n > 0

\vspace{1cm}

Beta函数和Gamma函数有非常紧密的关系:
\[{\color{blue} B(x,y) = \frac{\Gamma(x)\Gamma(y)}{\Gamma(x+y)}}\]
\textit{
对\(\Gamma\)函数动手,令\(t=x^2\),分别得到
\[ \Gamma(p) = \int_{0}^{\infty} t^{p-1}e^{-t}\trm{d}t = \int_{0}^{\infty} x^{2(p-1)}e^{-x^2}(2x)\trm{d}x = 2\int_{0}^{\infty} x^{2p-1}e^{-x^2}\trm{d}x\]
因此
\[ \Gamma(p)\Gamma(q) = 4\int_{0}^{\infty} x^{2p-1}e^{-x^2}\trm{d}x\int_{0}^{\infty} y^{2q-1}e^{-y^2}\trm{d}y = 4 \iint_{\mathbb{R}_+^2} x^{2p-1}y^{2q-1}e^{-(x^2+y^2)} \trm{d}x\trm{d}y\]
用极坐标换元,再拆成两个积分
\[ \mbox{原式}=4\int_{0}^{\frac{\pi}{2}} \trm{d}\theta \int_{0}^{\infty} e^{-\rho^2}\rho^{2p+2q-2}\cos^{2p-1}\theta\sin^{2q-1}\theta\cdot\rho\trm{d}\rho = 4\int_{0}^{\frac{\pi}{2}} \cos^{2p-1}\theta\sin^{2q-1}\theta \trm{d}\theta \int_{0}^{\infty} e^{-\rho^2}\rho^{2p+2q-1} \trm{d}\rho\]
左边的积分即为\(\frac{1}{2}\trm{B}(p,q)\);令\(t=x^2\)后,右边的积分即为\(\frac{1}{2}\Gamma(p+q)\),即
\[ \Gamma(p)\Gamma(q) = \trm{B}(p,q)\Gamma(p+q)\]
}

Beta函数有如下性质:\\
(1) 对称性
\[ \trm{B}(x,y) = \trm{B}(y,x) \]
(2) 帕斯卡恒等式(Pascal's identity)的某种变形
\[ \trm{B}(x,y) = \trm{B}(x,y+1)+\trm{B}(x+1,y)\]
(3) 递推公式(复现律)(recurrence rule)
\[ \trm{B}(x+1,y) = \trm{B}(x,y)\cdot\frac{x}{x+y}\]
(4) 对于其中一个自变量的余元公式
\[ \trm{B}(x,1-x) = \frac{\pi}{\sin(\pi x)} \]
(5)
\[ \trm{B}(x,y)\cdot\trm{B}(x+y,1-y) = \frac{\pi}{x\sin(\pi y)}\]
(6) 与组合数的关系
\[ (x+1) \begin{pmatrix} x \\ y \end{pmatrix} = \frac{1}{B(x-y+1,y+1)}\]

当\(x,y\)都很大时,根据斯特林公式,\(\displaystyle\mathrm{B}(x,y)\sim \sqrt{2\pi}\frac {x^{x-\frac{1}{2}}y^{y-\frac{1}{2}}}{(x+y)^{x+y-\frac{1}{2}}}\),如果只是\(x\)很大时,那么可以用Gamma函数估计:\(\trm{B}(x,y)\sim\Gamma(y)x^{-y}\).

\subsection{点火公式}

点火公式是民间称呼,它指的是这个公式
\begin{theorem}{华里士积分(Wallis integeral)}
    \[ W_n = \int_{0}^{\frac{\pi}{2}}\sin^n(x)\trm{d}x = \left\{\begin{aligned} & \frac{(n-1)!!}{n!!} & n\mbox{为奇数} \\ & \frac{(n-1)!!}{n!!}\frac{\pi}{2} & n\mbox{为正偶数}\end{aligned}\right.\]
\end{theorem}
经过简单换元,可以得到:
\begin{itemize}
    \item [(1)] \(\displaystyle{\int_{0}^{\frac{\pi}{2}}\cos^n(x)\trm{d}x = W_n}\)
    \item [(2)] \(\displaystyle{\int_{0}^{\pi}\sin^n(x) = 2W_n}\)
    \item [(3)] \(\displaystyle{\int_{0}^{\pi}\cos^n(x) = \left\{\begin{aligned} & 0 & n\mbox{为正奇数} \\ & 2W_n & n\mbox{为正偶数}\end{aligned}\right.}\)
    \item [(4)] \(\displaystyle{\int_{0}^{2\pi}\sin^n(x) = \int_{0}^{2\pi}\cos^n(x) = \left\{\begin{aligned} & 0 & n\mbox{为正奇数} \\ & 4W_n & n\mbox{为正偶数}\end{aligned}\right.}\)
\end{itemize}
其中双阶乘定义为
\[ n!! = \left\{
    \begin{aligned} 
        & n \times (n-2) \times (n-4) \times \cdots \times 3 \times 1 & n\mbox{为正奇数} \\ 
        & n \times (n-2) \times (n-4) \times \cdots \times 4 \times 2 & n\mbox{为正偶数}
    \end{aligned}
\right.\]

\vspace{1cm}

\textit{
以前证明点火公式都是使用分部积分求出\(\{W_n\}\)的递推公式,再求出\(W_1\)然后解出通项公式,具体如下
\[W_n = \int_{0}^{\frac{\pi}{2}}(\sin^{n-2}x)(1-\cos ^{2}x)\trm{d}x = \int_{0}^{\frac{\pi}{2}}\sin^{n-2}x\trm{d}x-\int_{0}^{\frac{\pi}{2}}\sin^{n-2}x\cos^{2}x\trm{d}x\]
其中
\[\int_{0}^{\frac{\pi}{2}}\sin^{n-2}x\cos^{2}x\trm{d}x=\left.{\frac{\sin^{n-1}x}{n-1}}\cos x\right|_{0}^{\frac{\pi}{2}}+{\frac{1}{n-1}}\int_{0}^{\frac{\pi}{2}}\sin^{n-1}x\sin x\trm{d}x=0+{\frac{1}{n-1}}W_{n}\]
这就求出了\(W_n = W_{n-2} - \dfrac{W_{n}}{n-1}\),接着又因为\(W_0=1, W_1=\dfrac{\pi}{2}\),就求出了\(\{W_n\}\)的通项公式。
}

\vspace{1cm}

现在可以发现,华里士积分实际上是\(B\)函数等价表达式的特殊情况:
\[W_n = \int_{0}^{\frac{\pi}{2}}\sin^nx\cos^0x\trm{d}x = \int_{0}^{\frac{\pi}{2}}\sin^{2\frac{n+1}{2}-1}x\cos^{2(\frac{1}{2})-1}x = \frac{1}{2}B\left(\frac{n+1}{2},\frac{1}{2}\right) = \frac{\Gamma(\frac{n+1}{2})\Gamma(\frac{1}{2})}{2\Gamma(\frac{n}{2}+1)}\]
根据这个关系,可以把华里士积分的\(n\)扩展到实数范围。接着再分类讨论\(\Gamma\)函数:\\
(1) 若\(n\)是奇数,则\(\dfrac{n+1}{2}\)是整数,\(\dfrac{n}{2}+1\)则不是,因此
\[W_n = \frac{\Gamma(\frac{n+1}{2})\Gamma(\frac{1}{2})}{2\Gamma(\frac{n}{2}+1)} = \frac{\frac{n-1}{2}!\sqrt{\pi}}{2\sqrt{\pi}\frac{n!!}{2^{(n+1)/2}}} = \frac{\frac{n-1}{2}!!2^{\frac{n-1}{2}}}{n!!} = \frac{(n-1)!!}{n!!}\]
(2) 若\(n\)是偶数,则\(\dfrac{n+1}{2}\)不是整数,\(\dfrac{n}{2}+1\)则是,因此
\[W_n = \frac{\Gamma(\frac{n+1}{2})\Gamma(\frac{1}{2})}{2\Gamma(\frac{n}{2}+1)} = \frac{\frac{(n-1)!!}{2^{n/2}}\sqrt{\pi}\cdot\sqrt{\pi}}{2\frac{n}{2}!!} = \frac{\pi}{2}\cdot\frac{(n-1)!!}{n!!}\]
这就导出了点火公式。

把点火公式拓展一下,推导方法同上,可以得到:
\[ \int_{0}^{\frac{\pi}{2}} \sin^px\cos^qx\trm{d}x = \frac{1}{2}B\left(\frac{p+1}{2},\frac{q+1}{2}\right) = \left\{ \begin{aligned} & \frac{\pi}{2}\cdot\frac{(p-1)!!(q-1)!!}{(p+q)!!} & p,q\mbox{都为偶数} \\ & \frac{(p-1)!!(q-1)!!}{(p+q)!!} & \mbox{其他} \end{aligned}\right.\]

\vspace{1cm}

接下来讨论一下\(W_n\)的增长趋势. 我们知道,在区间\(\displaystyle{[0,\frac{\pi}{2}]}\)上,\(\sin(x)\leq x\). 并且由于\(\sin(x)\)在该区间上递增,所以有\(\sin(\sin(x)) \leq \sin(x)\),照此下去,得到更一般的\(\sin^{n+1}(x) \leq \sin^{n}(x)\),根据定积分的性质,\(W_n\)是递减的数列.

还没完,根据以上关系可以推知
\begin{align*}
    & \sin^{2n+1}(x) \leq \sin^{2n}(x) \leq \sin^{2n-1}(x) \\
    \Longrightarrow \quad & W_{2n+1}<W_{2n}<W_{2n-1} \\
    \Longrightarrow \quad & 1 \leq \frac{W_{2n}}{W_{2n+1}} \leq \frac{W_{2n-1}}{W_{2n+1}} = \frac{2n-1}{2n}\\
\end{align*}

根据夹逼准则,取极限得到\(\displaystyle{\lim_{n \to \infty} \frac{W_{2n}}{W_{2n+1}} = 1 }\),然而
\[\lim_{n \to \infty} \frac{W_{2n}}{W_{2n+1}} = \lim_{n \to \infty} \frac{\pi}{2} \frac{(2n-1)!!(2n+1)!!}{(2n)!!(2n)!!}\]
其中
\begin{align*}
    (2n)!! &= 2 \times 4 \times 6 \times \cdots \times (2n) = \prod_{k=1}^{n} 2k \\
    (2n+1)!! &= 3 \times 5 \times 7 \times \cdots \times (2n+1) = \prod_{k=1}^{n} (2k+1) \\
    (2n-1)!! &= 1 \times 3 \times 5 \times \cdots \times (2n-1) = \prod_{k=1}^{n} (2k-1)
\end{align*}
所以上式可以写为
\[\lim_{n \to \infty} \frac{W_{2n}}{W_{2n+1}} = \frac{\pi}{2} \lim_{n \to \infty}  \prod_{k=1}^{n} \left(\frac{2k+1}{2k}\cdot\frac{2k-1}{2k}\right) = 1\]
整理一下,就得到了
\begin{theorem}{华里士乘积(Wallis product)}
    \[\frac{\pi}{2} = \prod_{n=1}^{\infty} \left(\frac{2n}{2n+1}\cdot\frac{2n}{2n-1}\right) = \left(\frac{2}{1}\cdot\frac{2}{3}\right)\left(\frac{4}{3}\cdot\frac{4}{5}\right)\left(\frac{6}{5}\cdot\frac{6}{7}\right)\cdots\]
\end{theorem}

还有更简单的推导方法,直接代\(z=1/2\)到余元公式即可
\[\Gamma^2\left(\frac{1}{2}\right) = \pi = -\frac{1}{2}\cdot\frac{1}{(1/2)^2}\prod_{n=1}^{\infty}\left(1-\frac{(1/2)^2}{n^2}\right)^{-1} = 2\prod_{n=1}^{\infty}\left(\frac{2n}{2n+1}\cdot\frac{2n}{2n-1}\right)\]

\subsection{高斯积分}

上面的式子\(\displaystyle{\Gamma^2\left(\frac{1}{2}\right) = \pi}\)告诉我们
\[\sqrt{\pi} = \Gamma\left(\frac{1}{2}\right) := \int_{0}^{\infty}t^{-1/2}e^{-t}\trm{d}t = 2\int_{0}^{\infty}e^{-t}\trm{d}\sqrt{t} \xlongequal[]{x=\sqrt{t}} = 2\int_{0}^{\infty}e^{-x^2}\trm{d}x\]
考虑到被积函数是偶函数,所以这就得到了高斯积分
\begin{theorem}{高斯积分 或 欧拉-泊松积分}
    \[\int_{-\infty}^{\infty}e^{-x^2}\trm{d}x = \sqrt{\pi}\]
\end{theorem}
还记得以前是怎样求解这个积分的吗?
\begin{align*}
    \int_{-\infty}^{+\infty} e^{-x^2} \trm{d} x &= \sqrt{\int_{-\infty}^{+\infty} e^{-x^2} \trm{d}x \cdot \int_{-\infty}^{+\infty} e^{-y^2} \trm{d}y} \\
    &= \sqrt{\int_{-\infty}^{\infty}\int_{-\infty}^{\infty} e^{-(x^2+y^2)} \trm{d}x \trm{d} y} = \sqrt{\int_{0}^{2 \pi}\int_{0}^{+\infty} e^{-\rho^2} \rho\trm{d} \rho \trm{d} \theta} = \sqrt{\pi}
\end{align*}
更一般地,有
\begin{itemize}
    \item [(1)] \(\displaystyle{\int_{-\infty}^{\infty}e^{-a(x+b)^2} = \sqrt{\frac{\pi}{a}}}\)
    \item [(2)] \(\displaystyle{\int_{-\infty}^{\infty}ae^{-\frac{(x-b)^2}{2c^2}} = \sqrt{2\pi}a|c|}\)
    \item [(3)] \(\displaystyle{\int_{-\infty}^{\infty}e^{-ax^2+bx+c} = \sqrt{\frac{\pi}{a}}}e^{\frac{b^2}{4a}+c}\)
\end{itemize}

容易发现,上面这些积分和正态分布函数有点像. 确实,可以定义一系列函数
\begin{definition}{正态分布函数及其衍生}
    \begin{itemize}
        \item [(1)] 误差函数(error function)
        \[ \trm{erf}(z):=\frac{2}{\sqrt{\pi}}\int_{0}^{z}e^{-t^{2}}dt \]
        \item [(2)] 互补误差函数(complementary error function)
        \[ \trm{erfc}(z) := 1-\trm{erf}(z) = \frac{2}{\sqrt{\pi}}\int_{z}^{+\infty}e^{-t^{2}}dt \]
        \item [(3)] 虚误差函数(imaginary error function)
        \[ \trm{erfi}(z) := -i\trm{erf}(iz) = \frac{2}{\sqrt{\pi}}\int_{0}^{iz}e^{-t^{2}}dt \]
        \item [(4)] 标准正态分布函数(standard normal CDF)
        \[ \Phi(x) := \frac{1}{\sqrt{2\pi}} \int_{-\infty}^{x} e^{-\frac{t^2}{2}} \trm{d}t \]
        \item [(5)] 正态分布右尾函数(standard normal CCDF)
        \[ Q(x) := 1-\Phi(x) = \frac{1}{\sqrt{2\pi}} \int_{x}^{+\infty} e^{-\frac{t^2}{2}} \trm{d}t \]
    \end{itemize}
\end{definition}

当\(x\ll1\)和\(x\gg1\)时,误差函数可以用不同的麦克劳林展开式逼近:
\begin{align*}
    \trm{erf}(z) &= \frac{1}{\pi}e^{-x^2} \sum_{k=0}^{\infty} \frac{(2x)^{2k+1}}{(2k+1)!!}, \quad x \ll 1 \\
    \trm{erf}(z) &\sim 1-\frac{e^{-x^2}}{\sqrt{\pi}}\sum_{k=0}^{\infty}\frac{(-1)^k(2k-1)!!}{2^k}x^{-(2k+1)}, \quad x \gg 1
\end{align*}

\subsection{斯特林公式}

之前提到,当\(n\)很大时,有
\[n! = \Gamma(n+1) \approx \sqrt{2\pi n}\left(\frac{n}{e}\right)^n\]
此即为\uline{斯特林近似公式}(Stirling's approximation),接下来简单推导一下,然后再用欧拉-麦克劳林公式得出精确的斯特林渐进展开式.
\newline
\textit{证明:}
\[ n! = \Gamma(n+1) = \int_{0}^{\infty}x^{n}e^{-x}\trm{d}x \xlongequal[\trm{d}x=n\trm{d}t]{x = nt} \int_{0}^{\infty}(nt)^{n}e^{-nt}n\trm{d}t = n^{n+1}\int_{0}^{\infty}t^{n}e^{-nt}\trm{d}x = n^{n+1}\int_{0}^{\infty}e^{n(\ln t - t)}\trm{d}t\]
\textit{接下来使用的技巧称为\uline{拉普拉斯方法}. 对于一个二阶可导的函数\(f(x)\),最值为\(x_0\),在最值处用泰勒公式展开}
\[f(x) = f(x_0)+f'(x_0)(x-x_0)+\frac{1}{2}f''(x_0)(x-x_0)^2+o(x^2)\]
\textit{由于是最值,所以\(f'(x_0)=0\),此时即为}
\[f(x) = f(x_0)+\frac{1}{2}f''(x_0)(x-x_0)^2+o(x^2)\]
\textit{所以}
\[\int e^{nf(x)}\trm{d}x \approx e^{nf(x_0)}\int e^{\frac{1}{2}nf''(x_0)(x-x_0)^2}\trm{d}x\]
\textit{在这里,\(f(x)=\ln x-x\),令\(\displaystyle{f'(x_0)=\left(\frac{1}{x}-1\right)_{x=x_0}=0}\),得到\(x_0=1\),于是}
\[n! = n^{n+1}\int_{0}^{\infty}e^{n(\ln t - t)}\trm{d}t \approx n^{n+1}e^{-n}\int_{0}^{\infty}e^{-\frac{1}{2}n(x-1)^2}\trm{d}x \xlongequal[\trm{d}x=\sqrt{2/n}\trm{d}u]{u^2=\frac{1}{2}n(x-1)^2} n^{n+1}e^{-n}\sqrt{\frac{2}{n}}\int_{0}^{\infty}e^{-u^2}\trm{d}u\]
\textit{注意到当\(n\)很大时,\(x<0\)时,\(e^{-\frac{1}{2}n(x-1)^2}\)非常接近\(0\),所以可以把积分区间扩展到整个\((-\infty, \infty)\):}
\[n! \approx n^{n+1}e^{-n}\sqrt{\frac{2}{n}}\int_{-\infty}^{\infty}e^{-u^2}\trm{d}u = n^{n+1}e^{-n}\sqrt{\frac{2\pi}{n}} = \sqrt{2\pi n}\left(\frac{n}{e}\right)^n\]
\textit{这就得到了最基本的斯特林近似公式.}

\subsection{分数阶导数和分数阶微积分}

我们学过\(\displaystyle{\frac{\trm{d}}{\trm{d}x}}\)表示什么意思,也知道\(\displaystyle{\frac{\trm{d^2}}{\trm{d}x^2}}\)表示什么意思,但我们可没学过\(\displaystyle{\frac{\trm{d^{\sqrt{2}}}}{\trm{d}x^{\sqrt{2}}}}\),这节就把微分的积分的阶数扩展到任意实数.

令
\[
    \left\{
        \begin{aligned} 
            I^0[f](x) &:= f(x) \\ 
            I^{n+1}[f](x) &:= \int_a^x I^n[f](t) \trm{d}t
        \end{aligned}
    \right.
\]
将前几项写出来,注意区分自变量和被积分的哑变量.
\begin{align*}
    I^0[f](x_0) &= f(x_0) \\
    I^1[f](x_1) &= \int_{a}^{x_1}f(x_0)\trm{d}x_0 \\
    I^2[f](x_2) &= \int_{a}^{x_2}\int_{a}^{x_1}f(x_0)\trm{d}x_0\trm{d}x_1 \\
    I^3[f](x_3) &= \int_{a}^{x_3}\int_{a}^{x_2}\int_{a}^{x_1}f(x_0)\trm{d}x_0\trm{d}x_1\trm{d}x_2 \\
    I^n[f](x_n) &= \int_{a}^{x_n} \cdots \int_{a}^{x_2}\int_{a}^{x_1}f(x_0)\trm{d}x_0 \trm{d}x_1 \cdots \trm{d}x_{n-1} 
\end{align*}
可见\(I^{n+1}\)把\(I^n\)的自变量当做哑变量做了一次变上限积分,而该上限是\(I^{n+1}\)的自变量. 现在我们就来计算这一系列\(I^n\)是多少. 首先计算\(I^2[f](x)\),使用二重积分换序的方法.
\[I^2[f](x_2) = \int_{a}^{x_2}\int_{a}^{x_1}f(x_0)\trm{d}x_0\trm{d}x_1 = \int_{a}^{x_2}\int_{x_0}^{x_2}f(x_0)\trm{d}x_1\trm{d}x_0 = \int_{a}^{x_2}f(x_0)(x_2-x_0)\trm{d}x_0\]
变量\(x_1\)消失了. 接下来计算\(I^3[f](x)\),注意自变量的变化
\begin{align*}
    I^3[f](x_3) &= \int_{a}^{x_3} {\color{blue} \int_{a}^{x_2}f(x_0)(x_2-x_0)\trm{d}x_0}\trm{d}x_2 
    = \int_{a}^{x_3} \int_{x_0}^{x_3} f(x_0)(x_2-x_0)\trm{d}x_2 \trm{d}x_0 \\
    &= \int_{a}^{x_3} f(x_0)\frac{(x_3-x_0)^2}{2}\trm{d}x_0
\end{align*}
变量\(x_2\)消失了. 接下来计算\(I^4[f](x_4)\),注意自变量的变化
\begin{align*}
    I^4[f](x_4) &= \int_{a}^{x_4} {\color{blue} \int_{a}^{x_3}f(x_0)\frac{(x_3-x_0)^2}{2}\trm{d}x_0}\trm{d}x_3
    = \int_{a}^{x_4} \int_{x_0}^{x_4} f(x_0)\frac{(x_3-x_0)^2}{2}\trm{d}x_3 \trm{d}x_0 \\
    &= \int_{a}^{x_4} f(x_0)\frac{(x_3-x_0)^3}{6}\trm{d}x_0
\end{align*}
根据以上结果,我们猜测:
\[I^n[f](x_n) = \int_{a}^{x_n} f(x_0)\frac{(x_n-x_0)^{n-1}}{(n-1)!}\trm{d}x_0\]
接下来用数学归纳法证明这个猜想,显然\(n=1\)的情况成立,所以接下来只用证明递推的阶段.
\begin{align*}
    I^{n+1}[f](x) &= \int_{a}^{x_{n+1}}{\color{blue}I^n[f](x_n)}\trm{d}x_{n+1} = \int_{a}^{x_{n+1}}{\color{blue}\int_{a}^{x_n}f(x_0)\frac{(x_n-x_0)^{n-1}}{(n-1)!}\trm{d}x_{0}}\trm{d}x_{n+1} \\
    &= \int_{a}^{x_{n+1}}\int_{x_0}^{x_{n+1}}f(x_0)\frac{(x_n-x_0)^{n-1}}{(n-1)!}\trm{d}x_{n+1}\trm{d}x_0 = \int_{a}^{x_{n+1}}f(x_0)\frac{(x_{n+1}-x_0)^n}{n!}\trm{d}x_0
\end{align*}
证毕. 这就是
\begin{theorem}{柯西迭代积分公式(Cauchy formula for repeated integration)}
    \[I^n[f](x) = \int_{a}^{x} \int_{a}^{x_{n-1}} \cdots \int_{a}^{x_2}\int_{a}^{x_1}f(x_0)\trm{d}x_0 \trm{d}x_1 \cdots \trm{d}x_{n-1} = \frac{1}{(n-1)!}\int_{a}^{x}f(t)(x-t)^{n-1}\trm{d}t\]
\end{theorem}

以上讨论了这么多,看起来就是随便定义了一个函数然后推导它的另一种表达式,和本节的主题似乎没有联系. 但假如现在令\(I^{-n}[f](x) = \dfrac{\partial^n}{\partial x^n}f(x)\),然后就可以发现\(\forall n,m \in \mathbb{Z}: I^{n}[I^{m}[f]](x) = I^{n+m}[f](x)\). 证明思路很简单,只要注意到变上限积分和导数是互逆运算即可. 如果将式中的阶乘推广为\(\Gamma\)函数,则可以把\(n\)的范围推广到任意实部为正的复数,而这就是
\begin{definition}{黎曼-刘维尔积分(Riemann-Liouville integral)}
    \[I^{\alpha}[f](x) = \frac{1}{\Gamma(\alpha)}\int_{a}^{x}f(t)(x-t)^{\alpha-1}\trm{d}t\]
\end{definition}

例如求\(f(x)\equiv 1\)的任意阶积分,为了方便,令积分下限\(a=0\).
\[I^{\alpha}[f](x) = \int_{0}^{x}(x-t)^{\alpha-1}\trm{d}t = \frac{x^{\alpha}}{\alpha\Gamma(\alpha)}\]
当\(\alpha=1/2\)时,得到\(I^{\alpha}[f](x) = 2\sqrt{\dfrac{x}{\pi}}\). \(f(x) \equiv 1\)积0次分得到的\(x\)的系数为\(0\),积1次分则为1,积半次则为\(1/2\),非常符合直觉.

\subsection{高维球的“体积”}

回顾欧式空间中圆和球的定义,它们都是到定点的距离为定值的点的集合,唯一的不同只体现在维度上.
\begin{itemize}
    \item[\(\bullet\)] 在二维平面,即\(xy\)直角坐标系下,定点在原点的圆的方程为\(x^2+y^2=r^2\),\(r\)称为圆的半径.
    \item[\(\bullet\)] 在三维空间,即\(xyz\)直角坐标系下,定点在原点的球的方程为\(x^2+y^2+z^2=r^2\),\(r\)称为球的半径.
    \item[\(\bullet\)] 由此可以推广到\(n\)维欧式空间中,在\(x_1x_2\cdots x_n\)的直角坐标系下,定点在原点的超球的方程为\(x_1^2+x_2^2+\cdots+x_n^2=r^2\),\(r\)为该超球的半径.
\end{itemize}

\end{document}
